Internet censorship is a phenomenon of societal importance and attracts
investigation from multiple disciplines.  Several research groups, such as Censored Planet, have deployed large scale Internet
measurement platforms to collect network
reachability data.  However, existing studies generally rely on manually
designed rules (i.e., using censorship fingerprints) to detect 
network-based Internet censorship
from the data.  While this rule-based approach yields a high true positive
detection rate, it suffers from several challenges: it requires human
expertise, is laborious, and cannot detect any censorship not captured by the
rules.  Seeking to overcome these challenges, we design and evaluate a classification model based on  latent
feature representation learning and an image-based classification model to detect
network-based Internet censorship. To infer latent feature representations from
network reachability data, we propose a sequence-to-sequence autoencoder to
capture the structure and the order of data elements in the data. To estimate
the probability of censorship events from the inferred latent features, we rely on a densely connected
multi-layer neural network model.  Our image-based classification model encodes a network reachability data
record as a gray-scale image and classifies the image as censored or not using a
dense convolutional neural network.  We compare and evaluate both approaches using data
sets from Censored Planet via a hold-out evaluation.  Both
classification models are capable of detecting network-based Internet
censorship as we were able to identify instances of
censorship not detected by the known fingerprints. Latent feature representations likely encode more nuances in the data since the latent feature learning approach discovers  a greater quantity, and a more
diverse set, of new censorship instances.

% Via a hold-out evaluation, we train the models
% using the data records labeled using known censorship fingerprints. Our results
% show that although both approaches have alike predictive performance on the
% data records that can be clearly distinguished as censored or uncensored using
% known censorship fingerprints, the detection model using latent feature
% representations yields far more instances of candidate censorship events from
% undetermined data records than the image-based model. 
% Examining these evidence,
