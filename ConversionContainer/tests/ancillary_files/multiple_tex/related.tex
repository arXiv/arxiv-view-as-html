\section{Related Work}
\label{sec:related}

\paragraph{Internet Censorship} 
Internet Censorship has been a global phenomenon, which has attracted a
sustained research interest and resulted in a plethora of literature.  
%
A significant body of the literature is to understand the prevalence of
censorship, such as, to measure the online content, the network protocols and
applications, and the geographical regions that are subject to
censorship~\cite{aryan2013internet, nabi2013anatomy, singh2017characterizing,
yadav2018light, ng2018detecting,
meserve2018google, bock2021even, padmanabhan2021multi, ververis2021understanding}.  
%
Another is to unpack the censorship mechanisms, such as,
IP-based blocking, TCP packet injection, Transparent proxy,  DNS manipulation,
and SNI filtering~\cite{aceto_internet_2015, chai2019importance, bock2021even,
niaki2020iclab}.  

This work is not aimed to expand these bodies of knowledge.  Instead, our focus
is about the design of the ``end-to-end'' network-based Internet censorship
detection system by leveraging the recent advances in deep learning and a wealth
of network reachability data collected by the large scale Internet measurement
platforms. 


\paragraph{Censorship Measurement Platforms}
Large scale remote network measurement
platforms for studying Internet censorship have emerged over the last decade.~\cite{niaki2020iclab,
filasto2012ooni, sundara_raman_censored_2020, pearce2017augur,
pearce2017global, scott2016satellite, vandersloot2018quack, raman_measuring_2020,
jin2021understanding}. These platforms fall into two broad categories with regard to the measurement
methods employed and the selection and use of vantage points.

Some platforms
like OONI~\cite{filasto2012ooni}, ICLab~\cite{niaki2020iclab}, and
Disguiser~\cite{jin2021understanding} use a mixture of volunteer hosts, 
selected VPN servers, SOCKS Proxies, or RIPE Atlas servers as vantage 
points. From these vantage points the platforms make attempts to access a brand range
of content and services on the Internet, typically {\em like an
ordinary user would do}, and collect reachability data.  

Platforms like Censored
Planet~\cite{sundara_raman_censored_2020} take a different approach. Using systems like
Augur~\cite{pearce2017augur}, Iris~\cite{pearce2017global},
Satellite~\cite{scott2016satellite}, Quack~\cite{vandersloot2018quack}, and
HyperQuack~\cite{raman_measuring_2020}, these platforms rely on {\em network side-channels} to
gauge potential network censorship events. The distinction here is that
ordinary users would not access online content and services via these
side-channels; however, the platforms can select servers that support the
side-channel measurement mechanism globally as the vantage points.  These two
categories of platforms are complementary. For instance, Censored Planet can
select far more vantage points to provide a more complete geographical and
network coverage and continuity than OONI and
ICLab~\cite{sundara_raman_censored_2020}; however, the censorship mechanisms
that are applied to ordinary network traffic may not be always applicable to
the network side-channels, and as such, Censored Planet may not be able to
detect some Internet censorship events, which can lead to higher false negative
rate than OONI and ICLab~\cite{niaki2020iclab}.  

The platforms are the foundation and the motivation of this work. First, these
platforms are continuously collecting network measurement data, which provides
a foundation for the proposed data-driven Internet
censorship detection systems using deep neural networks. Second, the platforms have made available 
a large collection of censorship fingerprints, e.g., Censored Planet makes
available a tool to scan the collected data records to match a set of known
censorship fingerprints. Our system adopts a supervised learning model to
classify network reachability data records, and for this the fingerprints and
the tool are particularly useful to create labeled training data set 
for the classification model.
Third, some platforms use machine learning, particularly, unsupervised
learning to cluster network reachability data records to design new censorship
fingerprints~\cite{raman_measuring_2020, niaki2020iclab}, which is a 
laborious process and requires human expertise. This
observation motivates us to propose the end-to-end network-based
Internet censorship detection systems. 

For this work, we elect to use the Censored Planet data set to study the
proposed approaches to detect network-based Internet censorship.  Censored Planet
is distinct in that the
platform collects network measurement data on network side-channels.  The
censorship measured is therefore network-based censorship other than
host and server-based censorship~\cite{aceto_internet_2015} --- an exception may 
be censorship mechanism deployed on host-based network filters or firewalls.  

\paragraph{Machine Learning for Censorship Detection}
Our work is a machine learning-based approach for censorship detection. There are
studies that propose machine learning approaches for censorship
detection. For instance, a study uses machine learning to understand Chinese
social media censorship~\cite{li2015predicting}.  However, these approaches are
for the host-based or the server-based censorship because the data used in
their studies are the content of the blogs being censored.  Different from
those, our work is on network-based Internet censorship detection in that the
data on which our approaches rely are network measurement data and do not include
the online content being censored.

Additionally, machine learning has been used to design sophisticated
Internet censorship methods or to recognize certain types of network
traffic~\cite{gao2021machine, trivedi2016fully}. This trend adds urgency to studies
on machine-learning based censorship detection like this one. This necessity of this
work is increased given that the literature on detecting
network-based Internet censorship using machine learning is still rare.


\paragraph{Learning from Network Measurement Data}
This work is about learning from the network reachability data collected by
censorship measurement platforms to detect Internet censorship. The network
reachability data is a type of network measurement data. A deep learning model
has been used to characterize Internet hosts. In particular, a variational
autoencoder, an unsupervised neural network model, was previously designed to construct
low-dimensional embeddings of the high-dimensional binary representations of
the IP intelligence data collected by Censys, a type of network measurement
data~\cite{sarabi_characterizing_2018}.  Our work significantly
benefits from this and in spirit is an application of such an approach. 
However, there are
several differences.  First, the problem we investigate, network-based
Internet censorship detection, is distinct from host characterization. Second, our
approach to infer the low-dimensional embeddings is a sequence-to-sequence
model, which is motivated by the observation that the order of data elements
matters. Third, we compare the latent feature representation approach with a
state-of-the-art image-based approach. Finally, we design an engineering
solution to overcome the hurdle that the censorship network 
measurement data are enormous.  




% Our work is about the design of an ``end-to-end'' system by adopting recent
% machine learning technique. Related to this is the work that use machine
% learning for network measurement data. 


% An emerging line of work adopts remote measurement techniques to study Internet censorship.
% 
% 
% 
% 
% 
% and to measure the prevalence of censorship, such
% as .
% 
% 
% Content-based censorship detection -- require access to content. 
% Difficult to observe from unharmful manner.
% 
% 
% Machine learning based approach
% 
% 
% Censored planet using clustering (features). 
% 
% 
% The researchers at Censored Planet have been processing their data
% using filtering scripts based on features defined by their knowledge
% of the censorship domain. Central to their approach has been a
% set of "block page fingerprints" which are strings of text found in
% the HTML of known block pages. [5] Responses are processed via
% regular expressions to identify matches with these strings. This
% fingerprint matching process has been the primary means of finding
% censored responses. Recently, the Censored Planet team also began
% rendering the HTML response into an image, and using an image
% classifying neural network, ResNet50, to cluster responses. [20]
% They then examine a sample of each cluster to verify block pages,
% identify new pages and engineer a new fingerprint.
% The Censored Planet researchers intentionally chose the image of
% a block page for processing through ResNet50 because block pages
% from the same tool can be constructed in various languages. From
% the researchers’ perspective the similarity of visual design allows
% them to ignore the language of the text. However, the deep learning
% models of Natural Language Processing would open the features
% of the text to be sources of insight. In addition, a visual inspection
% of the page even by a deep learning network does not include any
% data from the headers or other metadata surrounding the request
% and response.
% 
% 
